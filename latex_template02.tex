\documentclass[a4paper,12pt]{scrartcl}
%%%%%%%%%%%%%%%%%%%%%%%%%%%%%%%%%%%%%%%%%
% Bitte zuerst diese Variablen setzen!
%
\def\Nr{1}					%Hier die Nummer des Übungsblatts eintragen
\def\NameA{Alice Müller}	%Hier den Name eintragen
\def\MatA{1234567}			%Hier die Matrikelnummer eintragen
\def\NameB{Bob Meier}		%Hier den Name des 2. Gruppenmitglieds eintragen
\def\MatB{765421}			%Hier die Matrikelnummer des 2. Gruppenmitglieds eintragen
\def\NameC{}				%Name 3. Gruppenmitglied
\def\MatC{}					%Matrikelnummer 3. Gruppenmitglied
\def\NameD{}				%Name 4. Gruppenmitglied
\def\MatD{}					%Matrikelnummer 4. Gruppenmitglied
%
%%%%%%%%%%%%%%%%%%%%%%%%%%%%%%%%%%%%%%%%%
\usepackage[utf8]{inputenc}
\usepackage[ngerman]{babel}
\usepackage{hyperref}
\usepackage{booktabs}
\usepackage{geometry}
\usepackage{amssymb}
\usepackage{amsmath}
\usepackage{ifthen}
\usepackage{enumerate}
\usepackage{verbatim}
\usepackage{multicol}
\usepackage{algorithm}
\usepackage{xcolor}
\setlength{\marginparwidth}{2.5cm}
\usepackage{todonotes}
\usepackage{tikz}
\usepackage{forest}
\usetikzlibrary{
	arrows,
	arrows.meta,
	automata,
	calc,
	chains,
	trees,
	positioning,
	scopes,
	decorations.pathmorphing,
	shapes,
	backgrounds,
	chains,
	}

\geometry{margin=3cm, top=2.7cm}
\renewcommand{\thesection}{\arabic{section}{.}}

\begin{document}
\begin{center}
	\sffamily
	\bfseries
	\LARGE
	Datenstrukturen und Algorithmen\\
	\Large
	\vspace{.2\parskip}
	Hausübung \Nr\\
	\normalsize\normalfont
	WiSe 23/24
	\vspace{.2\parskip}
\end{center}

\begin{tabular}[t]{p{4.5cm} p{3cm}}
	\toprule
	Name & Matrikelnummer\\
	\midrule
	\NameA & \MatA\\
	\NameB & \MatB\\
	\ifx\NameC\empty\else\NameC & \MatC\\\fi
	\ifx\NameD\empty\else\NameD & \MatD\\\fi
	\bottomrule
\end{tabular}
\hfill
\begin{tabular}[t]{ccccc}
	\toprule
	A1 & A2 & A3 & Bonus & $\Sigma$ \\
	\midrule
	\\
	\bottomrule	
\end{tabular}
\hfill\\


%%%%%%%%%%%%%%%%%%%%%%%%%%%%%%%%%%%%%%%%%%%%%%%%%%%%%%%%%%%%%%%%%%%%%%%%%%%%%%%%%%%%%%%%%
% Ab hier bearbeiten 
%%%%%%%%%%%%%%%%%%%%%%%%%%%%%%%%%%%%%%%%%%%%%%%%%%%%%%%%%%%%%%%%%%%%%%%%%%%%%%%%%%%%%%%%%
Informationen zu \LaTeX{} auf z.B.: \url{https://tex.cloud.uni-hannover.de/learn}.

\section*{Aufgabe 1}
\begin{enumerate}[a)]
	\item
    $$
    \renewcommand{\arraystretch}{1.5}
    \begin{array}{|l|l|l|l|l|}
        \hline
        \text{ausgewählt} & \text{PriorityQueue} & \text{Knotenmenge} & \text{Gewicht} & \text{neue Kante}\\\hline
        & & & & \\\hline
        & & & & \\\hline
        & & & & \\\hline
        & & & & \\\hline
        & & & & \\\hline
        & & & & \\\hline
        & & & & \\\hline
        & & & & \\\hline
        & & & & \\\hline
    \end{array}
    $$
    \begin{center}
        \begin{tikzpicture}[
            baseline = (current bounding box.north),
            x=0.8cm,
            y=0.8cm,
            node/.style={draw=black,circle,minimum size=2em}
            ]
            \node[node] (a) {a};
            \node[node, below right = 2 of a] (d) {d};
            \node[node, above right = 2 of d] (b) {b};
            \node[node, below right = 2 of b] (e) {e};
            \node[node, above right = 2 of e] (c) {c};
            \node[node, below left  = 2 of d] (f) {f};
            \node[node, below right = 2 of d] (g) {g};
            \node[node, below right = 2 of e] (h) {h};
    
            \draw[-] (a) -- node[above=0mm] {7} (b);
            \draw[-] (b) -- node[above=0mm] {6} (c);
            \draw[-] (a) -- node[right=1mm] {2} (d);
            \draw[-] (b) -- node[right=1mm] {4} (d);
            \draw[-] (b) -- node[right=1mm] {2} (e);
            \draw[-] (c) -- node[right=1mm] {2} (e);
            \draw[-] (a) -- node[right=0mm] {4} (f);
            \draw[-] (c) -- node[right=0mm] {3} (h);
            \draw[-] (f) -- node[above=0mm] {5} (g);
            \draw[-] (g) -- node[above=0mm] {6} (h);
            \draw[-] (f) -- node[right=1mm] {3} (d);
            \draw[-] (d) -- node[right=1mm] {8} (g);
            \draw[-] (e) -- node[right=1mm] {4} (g);
            \draw[-] (e) -- node[right=1mm] {4} (h);
            \draw[-] (a) to[bend left, looseness=0.6] node[above=0mm] {5} (c);
            \draw[-] (f) to[bend right, looseness=0.6] node[below=0mm] {7} (h);
        \end{tikzpicture}
    \end{center}
    \item 
\end{enumerate}


\section*{Aufgabe 2}
\begin{enumerate}[a)]
	\item
	Graphen können mit dem Tikz-Paket erstellt werden\\
	\url{https://tex.cloud.uni-hannover.de/learn/latex/TikZ_package}\\
	\url{https://tobiw.de/tbdm/tikz-adventskalender}
    \begin{center}
        \begin{tikzpicture}[
            baseline = (current bounding box.north),
            x=0.8cm,
            y=0.8cm,
            node/.style={draw=black,circle,minimum size=2em}
            ]
            \node[node] (a) {a};
            \node[node, below right = 2 of a] (d) {d};
            \node[node, above right = 2 of d] (b) {b};
            \node[node, below right = 2 of b] (e) {e};
            \node[node, above right = 2 of e] (c) {c};
            \node[node, below left  = 2 of d] (f) {f};
            \node[node, below right = 2 of d] (g) {g};
            \node[node, below right = 2 of e] (h) {h};
    
            \draw[-] (a) -- node[above=0mm] {7} (b);
            \draw[-] (b) -- node[above=0mm] {6} (c);
            \draw[-] (a) -- node[right=1mm] {2} (d);
            \draw[-] (b) -- node[right=1mm] {4} (d);
            \draw[-] (b) -- node[right=1mm] {2} (e);
            \draw[-] (c) -- node[right=1mm] {2} (e);
            \draw[-] (a) -- node[right=0mm] {4} (f);
            \draw[-] (c) -- node[right=0mm] {3} (h);
            \draw[-] (f) -- node[above=0mm] {5} (g);
            \draw[-] (g) -- node[above=0mm] {6} (h);
            \draw[-] (f) -- node[right=1mm] {3} (d);
            \draw[-] (d) -- node[right=1mm] {8} (g);
            \draw[-] (e) -- node[right=1mm] {4} (g);
            \draw[-] (e) -- node[right=1mm] {4} (h);
            \draw[-] (a) to[bend left, looseness=0.6] node[above=0mm] {5} (c);
            \draw[-] (f) to[bend right, looseness=0.6] node[below=0mm] {7} (h);
        \end{tikzpicture}
    \end{center}
	\item
	\item
\end{enumerate}


\section*{Aufgabe 3}
\begin{enumerate}[a)]
	\item Liste: $[58, 3, 217, 13, 365, 524, 34, 134]$
	\item
	\item
\end{enumerate}
\end{document}